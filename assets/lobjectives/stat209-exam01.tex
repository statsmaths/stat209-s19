\documentclass[11pt]{article}
\usepackage[top=1.5cm,bottom=2cm,left=2cm,right= 2cm]{geometry}
\geometry{letterpaper}                   % ... or a4paper or a5paper or ...
%\geometry{landscape}                % Activate for for rotated page geometry
\usepackage[parfill]{parskip}    % Activate to begin paragraphs with an empty line rather than an indent
\usepackage{graphicx}
\usepackage{amssymb}
\usepackage{epstopdf}
\usepackage{amsmath}
\usepackage{multirow}
\usepackage{multicol}
\usepackage{changepage}
\usepackage{lscape}
\usepackage{enumitem}
\usepackage{ulem}
\DeclareGraphicsRule{.tif}{png}{.png}{`convert #1 `dirname #1`/`basename #1 .tif`.png}

\usepackage{xcolor}

\definecolor{oiB}{rgb}{.337,.608,.741}
\definecolor{oiR}{rgb}{.941,.318,.200}
\definecolor{oiG}{rgb}{.298,.447,.114}
\definecolor{oiY}{rgb}{.957,.863,0}

\definecolor{light}{rgb}{.337,.608,.741}
\definecolor{dark}{rgb}{.337,.608,.741}

\usepackage[colorlinks=false,pdfborder={0 0 0},urlcolor= dark,colorlinks=true,linkcolor=black]{hyperref}

\newcommand{\light}[1]{\textcolor{light}{\textbf{#1}}}
\newcommand{\dark}[1]{\textcolor{dark}{#1}}
\newcommand{\gray}[1]{\textcolor{gray}{#1}}

%\date{}                                           % Activate to display a given date or no date

%

\begin{document}

{\LARGE \textcolor{oiB}{Learning Objectives \hfill Exam 01: Two Categorical Variables}} \\

\begin{enumerate}
\renewcommand\labelenumi{\textcolor{light}{\textbf{LO \theenumi.}}}
\item Distinguish between the \textbf{explanatory variable(s)} and the
\textbf{response variable} in an experiment.
\begin{itemize}
\renewcommand{\labelitemi}{{\textcolor{dark}{{\tiny $\blacksquare$}}}}
\item \textbf{explanatory variable}: a variable that is either controlled
by the experimenter or assumed to be the cause of the outcome of the
experiment.
\item \textbf{response variable}: the primary variable of interest that is
believed to be affected by the values of the explanatory variables.
\end{itemize}
The designation of which variable is response or explanatory is often not
fixed, and may be a decision of the investigator.

\item Distinguish between \textbf{observational} and \textbf{experimental}
studies and the conclusions that can be derived from them.
\begin{itemize}
\renewcommand{\labelitemi}{{\textcolor{dark}{{\tiny $\blacksquare$}}}}
\item \textbf{observational study}: draws inferences from a sample to a
population where the explanatory variable is not under the control of the
researcher. This may be because of ethical concerns, logistical constraints,
or taking a \textit{sample of convenience}.
\item \textbf{experimental study}: draws inferences from a sample in which
the explanatory variable of interest is selected at random.
\end{itemize}
Casual relationships can only be identified through the use of experimental
studies.

\item Understand the role of the \textbf{null hypothesis} and
\textbf{alternative hypothesis} in inferential statistics.
\begin{itemize}
\renewcommand{\labelitemi}{{\textcolor{dark}{{\tiny $\blacksquare$}}}}
\item \textbf{Null hypothesis}: A statement about an unknown parameter, usually
that there is no relationship between two measured phenomena, or no association among groups.
Denoted by $H_{0}$.
\item \textbf{Alternative hypothesis}: A statement that directly contradicts the null
hypothesis. Denoted by $H_{A}$
\end{itemize}

\item Define a \textbf{test statistic} and its \textbf{sampling distribution} under
the null hypothesis ($H_{0}$). Identify the value of a test statistic within R's output
and as seen in scientific literature.

\item Understand how simulation can be used to produce a estimation of a sampling
distribution.

\item Describe the \textbf{p-value} in terms of a inferential statistics
\begin{itemize}
\renewcommand{\labelitemi}{{\textcolor{dark}{{\tiny $\blacksquare$}}}}
\item The \textbf{p-value} is the probability of observing a test statistics that is
\textit{at least as extreme} as the one observed.
\item A small p-value indicates strong support for the alternative hypothesis
($H_{A}$) in favor of the null hypothesis.
\item A large p-value does \underline{not} indicate that the null hypothesis
is likely to be true. Instead, it just indicates that there is no strong
evidence against it. Think of a court trial where the defendant is found
innocent.
\end{itemize}

\item Apply and interpret the use of the Z-test for differences in proportions
within \textbf{R}. Memorize code for producing the test output from a dataset. Identify
the null and alternative hypotheses, test statistic, p-value, and observed
odds ratio.

\item Understand the assumptions behind the Z-test for difference in proportions
and common alternatives:
\begin{itemize}
\renewcommand{\labelitemi}{{\textcolor{dark}{{\tiny $\blacksquare$}}}}
\item \textbf{Z-test for proportions} (\texttt{tmod\_z\_test\_prop}): assumes
that independent variables are selected beforehand and samples are independent
\item \textbf{Fisher's Exact Test} (\texttt{tmod\_chi\_squared\_test}):
assumes that the marginal sums of the contingency table are fixed and known
\item \textbf{Chi-squared Test} (\texttt{tmod\_fisher\_test}): assumes that
both categorical variables are randomly determined by the experiment
\end{itemize}
You should also understand that science and social science research commonly
conflates these tests and you should not put much faith on their `correct'
usage. Also, the results are generally not too far off between each of the
tests.

\item Install R, RStudio, and required packages on your laptop. (This will not
be on the exam)

\item Understand how to start a new R notebook, load libraries, and execute
code.

\item Organize tabular data using the \textbf{unit of observation}.

\item Produce a comma separated values (CSV) or Excel file with tabular data.

\item Memorize code for reading a dataset into R, including the relevant package
(either \texttt{readxl} or \texttt{readr}).

\item Memorize code for running an hypothesis test in R with the \texttt{tmodels}
package.

\item Follow general naming conventions when constructing variable names.
\begin{itemize}
\renewcommand{\labelitemi}{{\textcolor{dark}{{\tiny $\blacksquare$}}}}
\item only include lowercase letters, underscores, and numbers
\item do not start a variable name with a number of underscore
\item do not use spaces; replace with underscores when needed
\item keep names short and concise; do not add superfluous information
(for example, use \texttt{age} instead of \texttt{patient\_age},
\texttt{age\_years}, or \texttt{recorded\_age} unless you need to distinguish
two types of age variables)
\end{itemize}

% \item Understand the challenges of publication bias in scientific research and
% resulting replication crisis. Identify potential approaches for addressing
% these challenges, such as:
% \begin{itemize}
% \renewcommand{\labelitemi}{{\textcolor{dark}{{\tiny $\blacksquare$}}}}
% \item pre-registering results
% \item outlets for null results
% \item support for replication studies
% \item better informing journalists and public about challenges of
% over-generalizing and over-hyping a single study
% \end{itemize}

\end{enumerate}


\end{document}



% \item Compute and interpret the \textbf{odds ratio} (OR) of an outcome between two groups.
% \begin{itemize}
% \renewcommand{\labelitemi}{{\textcolor{dark}{{\tiny $\blacksquare$}}}}
% \item If an output occurs with probability $p_A$ in group $A$ and $p_B$ in
% group $B$, the \textbf{odds ratio} of the outcome is given by
% \begin{align*}
% \text{OR} &= \frac{p_A / (1 - p_A)}{p_B / (1 - p_B)}.
% \end{align*}
% If the OR is greater than 1, it is more likely to occur in group $A$. An
% OR less than 1 is more likely to occur in group $B$.
% \end{itemize}
