\documentclass[11pt]{article}
\usepackage[top=1.5cm,bottom=2cm,left=2cm,right= 2cm]{geometry}
\geometry{letterpaper}                   % ... or a4paper or a5paper or ...
%\geometry{landscape}                % Activate for for rotated page geometry
\usepackage[parfill]{parskip}    % Activate to begin paragraphs with an empty line rather than an indent
\usepackage{graphicx}
\usepackage{amssymb}
\usepackage{epstopdf}
\usepackage{amsmath}
\usepackage{multirow}
\usepackage{multicol}
\usepackage{changepage}
\usepackage{lscape}
\usepackage{enumitem}
\usepackage{ulem}
\DeclareGraphicsRule{.tif}{png}{.png}{`convert #1 `dirname #1`/`basename #1 .tif`.png}

\usepackage{xcolor}

\definecolor{oiB}{rgb}{.337,.608,.741}
\definecolor{oiR}{rgb}{.941,.318,.200}
\definecolor{oiG}{rgb}{.298,.447,.114}
\definecolor{oiY}{rgb}{.957,.863,0}

\definecolor{light}{rgb}{.337,.608,.741}
\definecolor{dark}{rgb}{.337,.608,.741}

\usepackage[colorlinks=false,pdfborder={0 0 0},urlcolor= dark,colorlinks=true,linkcolor=black]{hyperref}

\newcommand{\light}[1]{\textcolor{light}{\textbf{#1}}}
\newcommand{\dark}[1]{\textcolor{dark}{#1}}
\newcommand{\gray}[1]{\textcolor{gray}{#1}}

%\date{}                                           % Activate to display a given date or no date

%

\begin{document}

{\LARGE \textcolor{oiB}{Learning Objectives \hfill Exam 03: Inference with several variables}} \\

\begin{enumerate}
\renewcommand\labelenumi{\textcolor{light}{\textbf{LO \theenumi.}}}
\item Use R to compare the difference in mean across two groups using either
a parametric or non-parametric test.
\begin{itemize}
\renewcommand{\labelitemi}{{\textcolor{dark}{{\tiny $\blacksquare$}}}}
\item \textbf{two sample t-test}: a parametric test that relies on the central
limit theorem. Provides a confidence interval for the differences in the
means.
\item \textbf{Mann-Whitney U}: a non-parametric test for differences in the
mean. Does not include an effect size or confidence interval.
\end{itemize}

\item Use R to compare the difference in mean across more than two groups.
\begin{itemize}
\renewcommand{\labelitemi}{{\textcolor{dark}{{\tiny $\blacksquare$}}}}
\item \textbf{one-way ANOVA}: a parametric test that relies on the central
limit theory.
\item \textbf{Kruskal-Wallis test}: a non-parametric test to detect whether
samples from multiple groups come from the same distribution.
\end{itemize}

\item Understand the problem behind \textbf{multiple hypothesis testing}.

\item Define the \textbf{family-wise error rate} (FWER) and the \textbf{false
discovery rate} (FDR). Be able to control these with the Holm-Bonferroni
method or the Benjamini-Hochberg-Yekutieli (BHY) method.

\item Use R to find pairwise differences in means or proportions between
groups, correcting for multiple hypothesis testing.
\begin{itemize}
\renewcommand{\labelitemi}{{\textcolor{dark}{{\tiny $\blacksquare$}}}}
\item \textbf{Pairwise T-test}
\item \textbf{Pairwise Mann-Whitney U}
\item \textbf{Pairwise Proportions Test}
\end{itemize}

\item Understand the problem of confounding variables and the resulting bias
caused within the model.

\item Apply linear regression in R to describe a relationship with a numeric
dependent variable, potentially with one or more  confounding variables.

\item Understand the basic assumptions behind linear regression.
\begin{itemize}
\renewcommand{\labelitemi}{{\textcolor{dark}{{\tiny $\blacksquare$}}}}
\item The `true' mean of the dependent variable is a linear function of the
independent variables.
\item The observations are drawn independently from one another.
\item The variance of the dependent variable is unrelated to the observed
values of the independent variables.
\item The central limit theorem is used to make assumptions about the testing
statistics.
\end{itemize}

\item Apply logistic regression in R to describe a relationship with a
two-category dependent variable, potentially with one or more confounding
variables.

\item Map aesthetics within the grammar of graphics to influence the size,
color, shape, and opacity of the data points.

\end{enumerate}


\end{document}