\documentclass[11pt]{article}
\usepackage[top=1.5cm,bottom=2cm,left=2cm,right= 2cm]{geometry}
\geometry{letterpaper}                   % ... or a4paper or a5paper or ...
%\geometry{landscape}                % Activate for for rotated page geometry
\usepackage[parfill]{parskip}    % Activate to begin paragraphs with an empty line rather than an indent
\usepackage{graphicx}
\usepackage{amssymb}
\usepackage{epstopdf}
\usepackage{amsmath}
\usepackage{multirow}
\usepackage{multicol}
\usepackage{changepage}
\usepackage{lscape}
\usepackage{enumitem}
\usepackage{ulem}
\DeclareGraphicsRule{.tif}{png}{.png}{`convert #1 `dirname #1`/`basename #1 .tif`.png}

\usepackage{xcolor}

\definecolor{oiB}{rgb}{.337,.608,.741}
\definecolor{oiR}{rgb}{.941,.318,.200}
\definecolor{oiG}{rgb}{.298,.447,.114}
\definecolor{oiY}{rgb}{.957,.863,0}

\definecolor{light}{rgb}{.337,.608,.741}
\definecolor{dark}{rgb}{.337,.608,.741}

\usepackage[colorlinks=false,pdfborder={0 0 0},urlcolor= dark,colorlinks=true,linkcolor=black]{hyperref}

\newcommand{\light}[1]{\textcolor{light}{\textbf{#1}}}
\newcommand{\dark}[1]{\textcolor{dark}{#1}}
\newcommand{\gray}[1]{\textcolor{gray}{#1}}

%\date{}                                           % Activate to display a given date or no date

%

\begin{document}

{\LARGE \textcolor{oiB}{Learning Objectives \hfill Exam 02: Two Numeric Variables}} \\

\begin{enumerate}
\renewcommand\labelenumi{\textcolor{light}{\textbf{LO \theenumi.}}}
\item Classify variables as numeric or categorical. Further distinguish
between continuous and discrete numeric variables and
ordinal/unordered/free-form categorical variables.

\item Compute (by hand and with R) and interpret measurements of central values
of a numeric variable.
\begin{itemize}
\renewcommand{\labelitemi}{{\textcolor{dark}{{\tiny $\blacksquare$}}}}
\item \textbf{mean}: If we observe $n$ observations $x_1, \ldots, x_n$ of a
variable, the mean is defined as:
\begin{align*}
\bar{x} &= \frac{1}{n} \cdot \sum_i x_i.
\end{align*}
\item \textbf{median}: If we observe an odd number of observations of a
numeric variable the median is the \textit{middle value} that splits the data
into two halves. That is, the median $M$ is defined such that half of the data
is no bigger than $M$ and half of the data are no smaller than $M$. For an
observation with an even number of points, the median is the average of the
two middle values.
\end{itemize}

\item Compute (by hand and with R) and interpret measurements of variability
for a numeric variable.
\begin{itemize}
\renewcommand{\labelitemi}{{\textcolor{dark}{{\tiny $\blacksquare$}}}}
\item \textbf{variance}: If we observe $n$ observations $x_1, \ldots, x_n$ of a
variable, the variance is the average squared distance from the mean:
\begin{align*}
Var(x) &= \frac{1}{n - 1} \cdot \sum_i (x_i - \bar{x})^2.
\end{align*}
\item \textbf{standard deviation}: The standard deviation is the square-root
of the variance.
\begin{align*}
sd(x) &= \sqrt{Var(x)} = \sqrt{\frac{1}{n - 1} \cdot \sum_i (x_i - \bar{x})^2}.
\end{align*}
\end{itemize}

\item Visually identify the \textbf{normal distribution} and understand the
\textbf{central limit theorem (CLT)} through simulation studies. For us, the
CTL gives that for large same sizes, the sampling distribution of the mean
from an independent sample is approximately normal regardless of the original
distribution.

\item Define the \textbf{standard error}---the standard deviation of its
sampling distribution---and use it to quantify the uncertainty in a test
statistic.

\item Understand the meaning of the Pearson correlation coefficient and be
able to visually approximate the correlation from a scatter plot of data.

\item Run, using R, two types of correlation tests between two numeric
variables and interpret the results.
\begin{itemize}
\renewcommand{\labelitemi}{{\textcolor{dark}{{\tiny $\blacksquare$}}}}
\item \textbf{Pearson correlation coefficient} (r): relies on the central
limit theorem approximation
\item \textbf{Spearman's rank correlation coefficient} ($\rho$):
a non-parametric test that attempts to describe a relationship by a monotonic
function.
\item \textbf{Kendall rank correlation coefficient} ($\tau$):
a non-parametric test that attempts to measure how similar the \textit{rank}
order of two datasets are. Conceptually similar to Spearman's coefficent.
\end{itemize}

\item Describe the different assumptions that distinguish the Pearson correlation
test from the Spearman/Kendall correlation tests.

\item Understand the concept of \textbf{statistical power} and how this
relates to the choice between parametric and non-parametric inference tests.

\item Describe the definition of a \textbf{confidence interval} for a given
\textbf{confidence level} and relate the concept back to hypothesis testing.

\item Use the grammar of graphics to describe a data visualizations in R.
Specifically, produce a scatter plot (with an optional best-fit line) for
two variables from a given dataset or a histogram for a single numeric
variable.

\item Demonstrate the idea behind Simpson's paradox and explain how it can
be alleviated through visualizations and good experimental design.

\end{enumerate}


\end{document}