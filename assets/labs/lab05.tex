\documentclass{tufte-handout}

\usepackage{amssymb,amsmath}
% \usepackage{mathspec}
\usepackage{graphicx,grffile}
\usepackage{longtable}
\usepackage{booktabs}

\newtheorem{mydef}{Definition}
\newtheorem{thm}{Theorem}

\providecommand{\tightlist}{%
  \setlength{\itemsep}{0pt}\setlength{\parskip}{0pt}}

\setlength{\parindent}{0em}
\setlength{\parskip}{12pt}

\begin{document}

\justify

{\LARGE Lab 05}

\vspace*{18pt}

Download the \texttt{lab05.Rmd} file and open it using RStudio. Then, use the
R programming language to help you answer the questions below. \textbf{Don't
forget to fill out the worksheet form before the next class!}

For this lab we are going to look at a recent article (Published in
February 2018) titled ``Acute Lateral Ankle Sprain Prediction in Collegiate
Women's Soccer Players'' and published in the \textit{International Journal of
Sports Physical Therapy}. Open a copy of the article through the following
link:
\begin{center}
\texttt{https://www.ncbi.nlm.nih.gov/pmc/articles/PMC5808007/}
\end{center}
And answer the following questions (only some are about this paper).

\vspace*{12pt}

\textbf{1.} There are a number of statistical tests in this paper, but we are
going to restrict ourselves to the results in Table 3 and Table 4. Read the
abstract as best you can and look at the contingency table given in Table 3.
Would you classify this as an experimental or observational design? Why?

\textbf{2.} Read the conclusion of the paper. Do the authors correctly
interpret the results in terms of causality? In other words, do they
incorrectly assign a causal relationship under data collected through an
observational study?

\textbf{3.} Create a tabular dataset corresponding to the results in Table 3,
saving the file in a CSV or Excel format.

\textbf{4.} Read the dataset into R and run Fisher's Exact Test. What are the
null and alternative hypothesis in this statistical inference test? Does your
p-value match that given in the paper?

\textbf{5.} Now apply the chi-squared test and the Z-test for proportions.
How do these test p-values compare to the Fisher's Exact test?

\textbf{6.} Given the experimental design, which of the three tests that we
have actually seem the most applicable?

\textbf{7.} Repeat steps 3-5 for the results in Table 4. Do the tests give
reasonably similar p-values?

% \textbf{7.} Explain how publication bias may effect your ability to trust the
% results of this study.

% \textbf{9.} Think back to the first week of class. Can you recall one student
% dataset that was observational and one that was experimental? If not, try
% to devise a data collection scenario that matches the missing type.

\end{document}







